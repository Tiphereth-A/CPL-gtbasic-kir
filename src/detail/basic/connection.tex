\begin{frame}{连通 - 无向图}
	\only<1->{对于一张无向图 \(G = (V, E)\), 对于 \(u, v \in V\), 若存在一条途径使得 \(v_0 = u, v_k = v\), 则称 \(u\) 和 \(v\) 是\textbf{连通的}}

	\only<2->{由定义, 任意一个顶点和自身连通, 任意一条边的两个端点连通}

	\only<3->{若无向图 \(G = (V, E)\), 满足其中任意两个顶点均连通, 则称 \(G\) 是\textbf{连通图}, \(G\) 的这一性质称作\textbf{连通性}}

	\only<4->{若 \(H\) 是 \(G\) 的一个连通子图, 且不存在 \(F\) 满足 \(H\subsetneq F \subseteq G\) 且 \(F\) 为连通图, 则 \(H\) 是 \(G\) 的一个\textbf{连通块/连通分量}(极大连通子图)}
\end{frame}


\begin{frame}{连通 - 有向图}
	\only<1->{对于一张有向图 \(G = (V, E)\), 对于 \(u, v \in V\), 若存在一条途径使得 \(v_0 = u, v_k = v\), 则称 \(u\) \textbf{可达}\xspace \(v\)}

	\only<2->{由定义, 任意一个顶点可达自身, 任意一条边的起点可达终点 (无向图中的连通也可以视作双向可达)}

	\only<3->{若一张有向图的结点两两互相可达, 则称这张图是\textbf{强连通}的}

	\only<4->{若一张有向图的边替换为无向边后可以得到一张连通图, 则称原来这张有向图是\textbf{弱连通}的}

	\only<5->{与连通分量类似, 也有\textbf{弱连通分量}(极大弱连通子图) 和\textbf{强连通分量}(极大强连通子图)}
\end{frame}

