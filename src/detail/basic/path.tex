\begin{frame}{路径}
	\begin{itemize}
		\item<1->\textbf{途径 (walk)}: 途径是连接一连串顶点的边的序列, 可以为有限或无限长度. 形式化地说, 一条有限途径 \(w\) 是一个边的序列 \(e_1, e_2, \ldots, e_k\), 使得存在一个顶点序列 \(v_0, v_1, \ldots, v_k\) 满足 \(e_i = (v_{i-1}, v_i)\), 其中 \(i \in [1, k]\)

		\only<2->{可以简写为 \(v_0 \to v_1 \to v_2 \to \cdots \to v_k\)}

		\only<3->{通常来说, 边的数量 \(k\) 被称作这条途径的\textbf{长度}}
		\item<4->\textbf{迹 (trail)}: 对于一条途径 \(w\), 若 \(e_1, e_2, \ldots, e_k\) 两两互不相同, 则称 \(w\) 是一条迹
		\item<5->\textbf{路径 (path)} / \textbf{简单路径 (simple path)}: 对于一条迹 \(w\), 若其连接的点的序列中点两两不同, 则称 \(w\) 是一条路径
		\item<6->\textbf{回路 (circuit)}: 对于一条迹 \(w\), 若 \(v_0 = v_k\), 则称 \(w\) 是一条回路
		\item<7->\textbf{环/圈 (cycle)} / \textbf{简单回路/简单环 (simple circuit)}: 对于一条回路 \(w\), 若 \(v_0 = v_k\) 是点序列中唯一重复出现的点对, 则称 \(w\) 是一个环
	\end{itemize}
\end{frame}
