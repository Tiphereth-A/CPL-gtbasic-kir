\begin{frame}{度矩阵, 邻接矩阵, 关联矩阵}
	\begin{alertblock}{注意}
		以下提到的图可以有重边, 但均不能有自环
	\end{alertblock}
\end{frame}


\begin{frame}[fragile]{度矩阵, 邻接矩阵, 关联矩阵}
	\begin{definition}
		\only<1->{对无向图 \(G=(V,E)\), 令 \(|V(G)|=n\), \(|E(G)|=m\)}

		\only<2->{\begin{itemize}
				\item<2-> 图\(G\) 的\textbf{度矩阵}\xspace \(D(G):=\operatorname{diag}\{d(i):i\in V(G)\}\)
				\item<3-> 图\(G\) 的\textbf{邻接矩阵}\xspace 定义为 \(n\) 阶方阵 \(A(G)\), 其中

					\begin{itemize}
						\item<4-> 边不带权时, \(A_{ij}(G)\) 一般表示点 \(i\) 和点 \(j\) 之间相连的边数

							若点 \(i\) 和点 \(j\) 不相邻则 \(A_{ij}(G)=0\)
						\item<5-> 边带权时, \(A_{ij}(G)\) 一般表示点 \(i\) 和点 \(j\) 之间相连的边的边权和

							若点 \(i\) 和点 \(j\) 不相邻则 \(A_{ij}(G)=\infty\)
					\end{itemize}
				\item<6-> 图\(G\) 的\textbf{关联矩阵}\xspace 定义为 \(n\times m\) 阶矩阵 \(M(G)\), 其中

					\begin{itemize}
						\item 若 \(M_{ij}(G)=1\), 则表示第 \(i\) 个点和第 \(j\) 条边相关联
						\item 若 \(M_{ij}(G)=0\), 则表示第 \(i\) 个点和第 \(j\) 条边不关联
					\end{itemize}
			\end{itemize}}
	\end{definition}
\end{frame}


\begin{frame}{度矩阵, 邻接矩阵, 关联矩阵}
	\includetikzimage{adjmat1.tex}{示例1}

	\[\begin{array}{cc}
			A=\begin{pmatrix}
				  0 & 1 & 0 & 1 \\
				  1 & 0 & 1 & 2 \\
				  0 & 1 & 0 & 1 \\
				  1 & 2 & 1 & 0
			  \end{pmatrix} &
			M=\begin{pmatrix}
				  1 & 0 & 0 & 1 & 0 & 0 \\
				  1 & 1 & 0 & 0 & 1 & 1 \\
				  0 & 1 & 1 & 0 & 0 & 0 \\
				  0 & 0 & 1 & 1 & 1 & 1
			  \end{pmatrix}
		\end{array}\]
\end{frame}


\begin{frame}{度矩阵, 邻接矩阵, 关联矩阵}
	\begin{definition}
		\only<1->{对有向图 \(G=(V,E)\), 令 \(|V(G)|=n\), \(|E(G)|=m\)}

		\only<2->{\begin{itemize}
				\item<2-> 图\(G\) 的\textbf{入度矩阵}\xspace \(D_{\text{in}}(G):=\operatorname{diag}\{d_{\text{in}}(i):i\in V(G)\}\),\textbf{出度矩阵}\xspace \(D_{\text{out}}(G):=\operatorname{diag}\{d_{\text{out}}(i):i\in V(G)\}\)
				\item<3-> 图\(G\) 的\textbf{邻接矩阵}\xspace 定义为 \(n\) 阶方阵 \(A(G)\), 其中

					\begin{itemize}
						\item<4-> 边不带权时, \(A_{ij}(G)\) 一般表示点 \(i\)\textbf{到}点 \(j\) 之间相连的边数

							若点 \(i\) 和点 \(j\) 之间没有边, 则 \(A_{ij}(G)=0\)
						\item<5-> 边带权时, \(A_{ij}(G)\) 一般表示点 \(i\)\textbf{到}点 \(j\) 之间相连的边的边权和

							若点 \(i\) 和点 \(j\) 之间没有边, 则 \(A_{ij}(G)=\infty\)
					\end{itemize}
				\item<6-> 图\(G\) 的\textbf{关联矩阵}\xspace 定义为 \(n\times m\) 阶矩阵 \(M(G)\), 其中

					\begin{itemize}
						\item 若 \(M_{ij}(G)=1\), 则表示第 \(i\) 个点是第 \(j\) 条边的终点
						\item 若 \(M_{ij}(G)=-1\), 则表示第 \(i\) 个点是第 \(j\) 条边的起点
						\item 若 \(M_{ij}(G)=0\), 则表示第 \(i\) 个点不是第 \(j\) 条边的端点
					\end{itemize}
			\end{itemize}}
	\end{definition}
\end{frame}


\begin{frame}{度矩阵, 邻接矩阵, 关联矩阵}
	\includetikzimage{adjmat2.tex}{示例2}

	\[\begin{array}{cc}
			A(G)=\begin{pmatrix}
				     0 & 0 & 0 & 1 \\
				     1 & 0 & 1 & 1 \\
				     0 & 0 & 0 & 0 \\
				     0 & 1 & 1 & 0
			     \end{pmatrix} &
			M(G)=\begin{pmatrix}
				     1  & 0  & 0  & -1 & 0  & 0  \\
				     -1 & -1 & 0  & 0  & -1 & 1  \\
				     0  & 1  & 1  & 0  & 0  & 0  \\
				     0  & 0  & -1 & 1  & 1  & -1
			     \end{pmatrix}
		\end{array}\]
\end{frame}


\begin{frame}[fragile]{度矩阵, 邻接矩阵, 关联矩阵}
	我们有一个简单的定理:

	\begin{theorem}
		\begin{itemize}
			\item 对无向图 \(G\),
			      \[
				      M(G)M^T(G)=D(G)+A(G)
			      \]
			\item 对有向图 \(G\),
			      \[
				      M(G)M^T(G)=D_{\text{in}}(G)+D_{\text{out}}(G)-A(G)
			      \]
		\end{itemize}
	\end{theorem}

	结合这三种矩阵的定义不难证明
\end{frame}
