\begin{frame}[fragile,allowframebreaks]{洛谷 P2144 {[}FJOI2007{]} 轮状病毒}
	\textbf{题目描述}

	轮状病毒有很多变种. 许多轮状病毒都是由一个轮状基产生. 一个 \(n\) 轮状基由圆环上 \(n\) 个不同的基原子和圆心的一个核原子构成. \(2\) 个原子之间的边表示这 \(2\) 个原子之间的信息通道, 如图 \ref{fig:lgp2144-1.tex}

	\includetikzimage{lgp2144-1.tex}{n 轮状基}

	\(n\) 轮状病毒的产生规律是在n轮状基中删除若干边, 使各原子之间有唯一一条信息通道. 例如, 共有 \(16\) 个不同的 \(3\) 轮状病毒, 如图 \ref{fig:lgp2144-2.tex} 所示

	给定 \(n(N\leq 100)\), 编程计算有多少个不同的 \(n\) 轮状病毒

	\includetikzimage{lgp2144-2.tex}{16 个 3 轮状病毒}

	\textbf{输入格式}

	第一行有 \(1\) 个正整数 \(n\)

	\textbf{输出格式}

	将编程计算出的不同的 \(n\) 轮状病毒数输出

	\textbf{样例输入}

	\includecode[common]{lgp2144.in}

	\textbf{样例输出}

	\includecode[common]{lgp2144.ans}
\end{frame}


\begin{frame}[fragile]{题解}
	\begin{equation}
		L(W_n) = \begin{bmatrix}
			n      & -1     & -1     & -1     & \cdots & -1     & -1     \\
			-1     & 3      & -1     &        & \cdots &        & -1     \\
			-1     & -1     & 3      & -1     & \cdots &        &        \\
			-1     &        & -1     & 3      & \cdots &        &        \\
			\vdots & \vdots & \vdots & \vdots & \ddots & \vdots & \vdots \\
			-1     &        &        &        & \cdots & 3      & -1     \\
			-1     & -1     &        &        & \cdots & -1     & 3
		\end{bmatrix}_{n+1}
	\end{equation}

	考虑计算 \(L(W_n)[1]\)
\end{frame}


\begin{frame}[fragile]{题解}
	令

	\[
		A_n = \begin{bmatrix}
			3  & -1 &        &        &        &    \\
			-1 & 3  & -1     &        &        &    \\
			   & -1 & 3      & \ddots &        &    \\
			   &    & \ddots & \ddots & \ddots &    \\
			   &    &        & \ddots & 3      & -1 \\
			   &    &        &        & -1     & 3
		\end{bmatrix}_{n}
	\]

	\(A_n\) 可以很容易地求得
\end{frame}


\begin{frame}[fragile]{三对角矩阵}
	\only<1->{称

	\[
		\begin{bmatrix}
			a_{11} & a_{12} &        &        &             &           \\
			a_{21} & a_{22} & a_{23} &        &             &           \\
			       & a_{32} & a_{33} & \ddots &             &           \\
			       &        & \ddots & \ddots & \ddots      &           \\
			       &        &        & \ddots & a_{n-1,n-1} & a_{n-1,n} \\
			       &        &        &        & a_{n,n-1}   & a_{nn}
		\end{bmatrix}_{n}
	\]

	为 \(n\) 阶三对角矩阵}

	\only<2->{一般的三对角矩阵行列式计算有 \(O(n)\) 算法}
\end{frame}


\begin{frame}[fragile]{三对角矩阵}
	\only<1->{特别的, 对形如

		\[
			A_n= \begin{bmatrix}
				a & b &        &        &        &   \\
				c & a & b      &        &        &   \\
				  & c & a      & \ddots &        &   \\
				  &   & \ddots & \ddots & \ddots &   \\
				  &   &        & \ddots & a      & b \\
				  &   &        &        & c      & a
			\end{bmatrix}
		\]

		的三对角矩阵, 有}

	\only<2->{\[
			\det A_n=a\det A_{n-1}-bc\det A_{n-2}
		\]}
\end{frame}


\begin{frame}[fragile]{题解}
	令

	\[
		A_n = \begin{bmatrix}
			3  & -1 &        &        &        &    \\
			-1 & 3  & -1     &        &        &    \\
			   & -1 & 3      & \ddots &        &    \\
			   &    & \ddots & \ddots & \ddots &    \\
			   &    &        & \ddots & 3      & -1 \\
			   &    &        &        & -1     & 3
		\end{bmatrix}_{n}
	\]

	回到本题, 我们有

	\begin{equation}
		\label{mtt:eq:an}
		\begin{cases}
			\det A_n = 3\det A_{n-1}-\det A_{n-2} \\
			\det A_1 = 3                          \\
			\det A_2 = 8
		\end{cases}
	\end{equation}
\end{frame}


\begin{frame}[fragile]{题解}
	类似地, 我们有

	\begin{equation}
		\label{mtt:eq:lwn1}
		\det L(W_n)[1] = 3\det A_{n-1}-2\det A_{n-2}-2
	\end{equation}

	将式 (\ref{mtt:eq:lwn1}) 和式 (\ref{mtt:eq:an}) 联立, 得

	\begin{equation}
		\label{mtt:eq:lwn2}
		\begin{cases}
			\det L(W_n)[1] = 3\det L(W_{n-1})[1]-\det L(W_{n-2})[1]+2 \\
			\det L(W_1)[1] = 1                                        \\
			\det L(W_2)[1] = 5
		\end{cases}
	\end{equation}

	\textbf{时间复杂度}\xspace 直接递推 \(O(n)\), 可加速到 \(O(\log n)\)
\end{frame}


\begin{frame}[fragile]{参考代码 (Python)}
	\includecode[py]{lgp2144.py}
\end{frame}


\begin{frame}[fragile]{One more thing}
	实际上我们可以求出 \(t(W_n)=\deg L(W_n)[1]\) 的通项公式

	\[
		t(W_n)=(\varphi^{2n}+\overline{\varphi}^{2n})-2
	\]

	其中 \(\varphi=\frac{1+\sqrt{5}}{2}\), \(\overline{\varphi}=\frac{1-\sqrt{5}}{2}\)
\end{frame}
